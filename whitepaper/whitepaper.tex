\documentclass{article}
\title{Conclave: Making Bitcoin Scale and be Useful}
\author{N. P. O'Donnell}
\begin{document}
\maketitle
\begin{abstract}
A second layer built atop BTC's base chain offering fast, cheap transactions including transactions to and from the 
base chain, even if the payee is offline. Deposited funds are locked in multi-signature outputs on the base chain while
micro-payment transactions happen on the second layer. Participants are rewarded with points for withdrawals they sign, 
and points take a long time to accumulate. Collected fees are regularly distributed to participants in proportion to 
their points. Participants caught behaving dishonestly lose all their points and must start again.
\end{abstract}

\section{Background}

Some argue the crypto-currency bubble of 2017, in which BTC saw its all-time-high of \$20k, burst because the transaction throughput limit of 7 TPS was reached. Whether this is true or not, few would argue that BTC's transaction throughput is satisfactory, given the ever-growing reach of BTC.

\section{Proposed Solution}

Conclave proposes to solve the scaling problem by re-implementing the Bitcoin protocol on a layer-two chain, and providing on and off ramps from the layer-one chain, enabling users to seamlessly transfer bitcoin inter-chain. A payer can send a transaction then go offline immediately, and payees do not need to be online when receiving funds. From a user perspective, the protocol is simple and lightweight, yet secure, and has more in common with Bitcoin's original ``post-and-forget" design, than complex interactive protocols such as LN. \\

\section{Proof-of-Trust}

\textit{Proof-of-trust} (PoT) is Conclave's primary security mechanism. Each participant in a PoT system has a positive real number associated with it called a \textit{trust score} (or simply \textit{trust}). Upon joining the PoT network, a new participant begins with a trust score of zero, which means they have not yet performed any actions which show evidence they are productive or trustworthy. Participants with a trust score greater than zero have shown some measure of positive, honest behaviour and have never \textit{been proven} to have behaved dishonestly. The greater a participant's trust score, the more trustworthy they are deemed to be.

\subsection{Earning Trust}

Upon joining a PoT system, a participant begins with a trust score of $0$. To earn trust, the participant must perform some useful actions, and proof of those actions must be freely available to all other participants to see, because trust scores are indepentantly calculated by all participants. \\

The Conclave protocol nominates just one action can which can earn a participant trust: signing \textit{withdrawal transactions} on the base chain. To be eligible to do this, the participant must first be one of the $m$ \textit{trustees} of a \textit{funding transaction}. Any participant who is a trustee of a withdrawal transaction can increase their trust score by signing an exit transaction when the opportunity arises do so. This motivates them to stay online for as long as possible. The trust score of a participant $N$ is calculated by the formula:

\[trust_N = \sum_{i=1}^n\frac{1}{t_i - t_0}\]
Where:
\begin{itemize}
	\item $n$ is the number of withdrawal transactions $N$ has signed in its lifetime.
	\item $t_i$ is the time when the $i$th UTXO for which $N$ was a trustee, was created.
	\item $t_0$ is the time of the inception transaction.
\end{itemize}

\subsection{Losing Trust}

A participant's trust score can stay the same or increase, but never decrease. If a participant breaks any of the \textit{trust rules}, it loses all trust points and is \textit{blacklisted}.

\subsection{Trust Rules}

The \textit{trust rules} are a set of rules which, if broken by a participant, result in that participant being blacklisted. Breaches of the trust rules include:
\begin{itemize}
	\item Spending a UTXO outside of the rules of the protocol.
	\item Sending provably false information to a peer or to a client.
	\item Attempting to claim a reward with a false \textit{proof-of-storage}.
\end{itemize}
	
\subsection{Community Signatures}

A community signature (CS) is a set of individual signatures of members of a PoT system, on a single message, which meets certain criteria:

\begin{enumerate}
	\item The summed trust of the signatories is at or above a certain bound. (\textit{MinTrust})
	\item The number of signatories is below a certain bound. (\textit{MaxSigners})
\end{enumerate}

The CS, together with the message it endorses is published to the bitcoin blockchain giving it an implicit timestamp. Since each community member's trust score is also purely derived from data on the blockchain, one can be certain about the validity of a CS once it's on the blockchain. The CS is valid if and only if it is valid \textit{at the position where it appears on the blockchain}.\\

\subsection{Blacklisting}

If a participant is blacklisted, his trust score is reduced to zero and may no longer participate in the PoT network. A blacklisted participant should not be named as a trustee for any future deposit transaction, but still may sign withdrawal transactions for which it was a trustee, but will earn no trust points. Once blacklisted, a participant can not be un-blacklisted.

\subsubsection{Self-Blacklisting}

\subsection{Non-Transferability of Trust}

\section{The Conclave Network}

\textit{The Conclave network} comprises the collection of participants (or nodes) and the interactions between them.

\subsection{Network Parameters}

The network parameters are a set of key-value data which affect the behaviour of the network.

\subsection{NPATs}

An \textit{NPAT} is a \textit{network parameter adjustment transaction}. 

\subsection{Inception Transaction}

The first NPAT is known as the \textit{inception transaction}. It gives the initial values of the network parameters. It also implicitly defines a zero trust score for all network participants. The CS parameters are also given initial values:

\begin{enumerate}
	\item \textit{MinTrust} is given an initial implicit value of zero.
	\item \textit{MaxSigners} is given an initial implicit value of infinity.
\end{enumerate}

\section{Transactions}

The term \textit{transaction} is somewhat overloaded in the world of blockchains. This is doubly true for Conclave since it manipulates two blockchains, each with their own transaction format. To describe how transactions work, I will begin by discussing the practical \textit{actions} which the user wishes to perform: \textit{deposits}, \textit{withdrawals}, and regular P2P payments, and explain how they are implemented in terms of \textit{transactions} on the Bitcoin and Conclave blockchains.

\subsection{Deposits}

A \textit{deposit} is where a user sends bitcoin from the Bitcoin network to the Conclave network. Each deposit involves exactly one transaction on the Bitcoin chain: the \textit{fund transaction} or \texttt{fundTx}, and exactly one transaction on the Conclave chain: the \textit{claim transaction} or \texttt{claimTx}. To make a deposit, a depositor pays a deposit amount $D$ into a P2WSH output, $O$, on the Bitcoin chain and some time later, may claim a value not exceeding $D$ on the Conclave chain, by broadcasting a claim transaction $C$ which proves that $O$ was broadcast with the intention of funding $C$ and $C$ only, and this decision was made before the fund transaction was broadcast. \\
			
			Functionally, a deposit is only concerned with one part of the fund transaction: the fund output, therefore it is possible for a single fund transaction to contain multiple fund outputs - each funding a different claim transaction. However a single claim transaction may only claim one fund output for itself. Each claim transaction includes a pointer to its fund output's location on the Bitcoin blockchain, known as its \textit{fundpoint}. A fundpoint is a standard ``outpoint" structure, consisting of a transaction ID and index. \\
			
			Before a claim transaction is broadcast claiming a particular fund output, the fund output will appear like any other P2WSH output. When the claim transaction is broadcast, together with the fundpoint, this reveals the fund output as such. When a node receives the claim transaction, it fetches the output located at the fundpoint, checks that it is a P2WSH output, then reconstructs the P2WSH redeem script known as the \textit{claim script}. The node then hashes the claim script and checks that the hash matches that of the fund output's \texttt{scriptPubKey}'s P2WSH script hash threby proving that the claim script, and thus the claim transaction, is the rightful claimant of the funds in the fund output. \\
			
			The structure of the claim script is as follows:
			
			\begin{center}
				\texttt{<claimTxHash> drop <n> <trustee 1 pubkey> ... <trustee n pubkey> <m>}
			\end{center}
			
			This is a standard multisignature script with the exception of the first two script elements, which push the \texttt{claimTxHash} on to the stack then pop it. The \texttt{claimTxHash} does not play any role in the logic of the script, however it needs to be there to cryptographically link the script to the claim transaction. Since the \texttt{claimTxHash} is a hash of all the fields of the claim transaction, the claim transaction can not be modified after the fund transaction is broadcast because then it would produce a different hash, breaking the cryptographic link with the fund output. \\
			
			There is no strict deadline imposed on a claimant for claiming a fund output, however if the fund output is very old by the time it is claimed, there may not be enough active trustees from the time it was created still on the network. If the fund output does not comply with the network's trust requirements \textit{at the time of claim}, it needs to be spent into a ``fresh" fund output before the claim is honoured. Once such an output is revealed to the network, the network will immediately try to spend it to a withdrawal transaction, as withdrawal transactions spend outputs in the order they appear in the blockchain, not the order in which they were claimed. However if the network cannot find $m$ active trustees to spend the output, it will be unspendable and the claim is lost.

			

\subsection{Withdrawals}

\subsection{``Regular" Transactions}

\section{Blockchain Design}

\subsection{Base Chain Tracking}

\subsection{Sharding Model}

``Store-as-much-as-you-can", describe math of keying

\subsection{Availability}

Permissionless nature, redundancy, data-for-sale

\subsection{Durability}

Proof-of-storage, incentive to store, redundancy

\subsection{Correctness}

Tx Verification, Fear of blacklisting

\subsection{Immutability}

\subsection{Consistency}

Metablocks, Hashchains

\section{Rewards}

\section{Proof-of-Storage}

\section{Possible Attacks}

Here we discuss some possible attacks against Conclave.

\subsection{Colluding Signers Attack}

A \textit{colluding signers attack} (CSA) is an attack where trustees entrusted to a Conclave-controlled output collude to spend the output outside the rules of the protocol. \\

CSAs can not be prevented but are strongly de-incentivised at the protocol level by ensuring that it makes no economic sense to partake in one. More precisely, we make sure no subset of $m$ trustees exists in the $n$ trustees of a deposit output such that the $m$ trustees would make less from their rewards than the value of the output in a short amount of time. \\

The ``short amount of time" is known as the \textit{contentment period}



\subsection{Sybil Attack}

A sybil attack is where a malicious actor creates a large number of false identities in order to gain a disproportionately large influence. Conclave nodes are identified by their public keys alone, so  a new identity can be created trivially by creating a new key pair, meaning many identities can be created at virtually zero cost. The degree to which a node is trusted is proportional to the node's trust score, and new nodes have zero trust scores so will not be trusted initially. No amount of zero-trust nodes can influence the network, therefore sybil attacks involving only zero-trust nodes are impossible. \\

But what about sybil attacks involving nodes with non-zero trust scores ? Building up trust is a time-consuming process, and a sybil attacker would need to do so for a number of nodes - enough to eventually atttack the network. A successful attack would entail multiple nodes controlled by a single attacker gaining enough trust to be able to create community signatures. Once this happens, the attacker could manipulate network parameters at will, effectively taking control of the network. \\

While this sounds like a plausable scenario, we need to consider the motivation behind such an attack, and what is likely to happen if such an attack were to occur. The primary motive behind most attacks on blockchains is money. If a sybil attacker's motive was money and they planned on building up enough trust among their nodes to steal a UTXO with a CSA, it's unlikely that this would be a financially sound choice compared with the alternative of behaving honestly. \textit{See the previous section for a more thorough explanation of CSAs}. \\

If the attacker's motivation is to take control of the network by making \textit{obviously bad} network parameter updates, these updates will quickly be noticed by vigilant observers and an emergency fix in the codebase which blacklists the attackers node IDs would be deployed by creating a hard-coded NPAT in the NPAT history which does not exist in the blockchain, fooling the node into thinking it's processed an NPAT just before the attacker's first attack NPAT. The hard-coded NPAT would blacklist the attacker's node IDs, causing the attacker's NPATs to become invalid. \\

This is a highly unlikely scenario but if it were to happen, we show here that recovery would be quick and straightforward.

\subsection{Lying}

All responses sent by network participants to both clients and peers must be signed together with the associated request. If the requestor thinks the responder has lied to them, the requestor may create and broadcast a TVR, which includes the signed request and response. If the community endorses that TVR, the response is thusly deemed a lie by the network and the responder is blacklisted. This motivates responders not to lie.

\section{Conclusion}

\end{document}
